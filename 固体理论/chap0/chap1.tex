\chapter{周期性结构}
\section{正空间和倒空间}
\subsection{仿射坐标系}
在空间中任取一原点O以及任意三个不共面的矢量$\boldsymbol{e}_1,\boldsymbol{e}_2,\boldsymbol{e}_3$,则可以构建一个空间仿射坐标系$\{O;\boldsymbol{e}_1,\boldsymbol{e}_2,\boldsymbol{e}_3\}$。空间任一矢量$\boldsymbol{a}$可以表示为:
\begin{equation}
    \boldsymbol{a} = a_1 \boldsymbol{e}_1 + a_2 \boldsymbol{e}_2 + a_3 \boldsymbol{e}_3 =a^i \boldsymbol{e}_i,
\end{equation}
其中,$a_1,a_2,a_3$称为空间仿射坐标。等式最右边则是用爱因斯坦求和约定的简洁形式。对于其他维度也类似,但考虑到实际问题,以下没有特殊说明,只考虑三维情况。
\subsection{欧式空间与度量系数}
对仿射空间赋予内积后,则称为欧式空间。即欧式空间是特殊的仿射空间。设$\boldsymbol{a},\boldsymbol{b}$是欧式空间的两个矢量,定义两者内积:
\begin{equation}
    \boldsymbol{a}\cdot \boldsymbol{b} = (a^i \boldsymbol{e}_i)\cdot (b^j \boldsymbol{e}_j) = 
    a^ib^j (\boldsymbol{e}_i \cdot \boldsymbol{e}_j) = g_{ij}a^ib^j,
\end{equation}
其中$g_{ij} = \boldsymbol{e}_i \cdot \boldsymbol{e}_j$称为欧式空间的度量系数。
\par 由于$\boldsymbol{a}\cdot \boldsymbol{b}  = \boldsymbol{b}\cdot \boldsymbol{a}$,容易验证度量系数有如下关系:
\begin{equation}
    g_{ji} = g_{ij}
\end{equation}
记度量系数$g_{ij}$构成的矩阵为g(我们称之为度量矩阵),则g为实对称矩阵,即
\begin{equation}
    \tilde{g} = g,
\end{equation}
其中$\tilde{g}$表示g的转置。
\par 有了内积的定义和度量系数后,我们来进行一些矢量的计算。首先考虑矢量$\boldsymbol{a}$与自身的点积:
\begin{equation}
    \boldsymbol{a^2} :=\boldsymbol{a}\cdot \boldsymbol{a} = g_{ij}a^i a^j,
\end{equation}
写成矩阵形式
\begin{equation}
    \boldsymbol{a}^2 = \left( 
    \begin{array}{ccc}
        a^1 & a^2 & a^3
    \end{array}\right)
    \left(\begin{array}{ccc}
         g_{11}&g_{12} &g_{13} \\
         g_{21}&g_{22} &g_{23} \\
         g_{31}&g_{32} &g_{33} \\
    \end{array}\right)
    \left(
    \begin{array}{c}
         a^1  \\
         a^2 \\
         a^3
    \end{array}\right)
\end{equation}
由于g是实对称矩阵,可以证明$\boldsymbol{a}^2 \ge 0$。
\par 现在我们考虑欧式空间三个基底$\boldsymbol{e}_1 ,\boldsymbol{e_2},\boldsymbol{e_3}$的混合积。任取一个笛卡尔坐标系,设其基矢为$\boldsymbol{i},\boldsymbol{j},\boldsymbol{k}$。设欧式空间三个基矢在笛卡尔坐标表示如下:
\begin{equation}
\begin{aligned}
        \boldsymbol{e}_1 = x_1 \boldsymbol{i} + y_1 \boldsymbol{j} + z_1 \boldsymbol{k} \\
    \boldsymbol{e}_2 = x_2 \boldsymbol{i} + y_2 \boldsymbol{j} + z_2 \boldsymbol{k} \\
    \boldsymbol{e}_3 = x_3 \boldsymbol{i} + y_3 \boldsymbol{j} + z_3 \boldsymbol{k} 
\end{aligned}
\end{equation}
因此,混合积在可以写成:
\begin{equation}
    (\boldsymbol{e}_1 \times \boldsymbol{e}_2) \cdot \boldsymbol{e}_3 = \left| \begin{array}{ccc}
        x_1 & y_1 & z_1  \\
        x_2 & y_2 & z_2  \\
        x_3 & y_3 & z_3
    \end{array}\right| = \left| \begin{array}{ccc}
        x_1 & x_2 & x_3  \\
        y_1 & y_2 & y_3  \\
        z_1 & z_2 & z_3
    \end{array}\right|
\end{equation}
于是,混合积平方:
\begin{equation}
    \begin{aligned}
        \relax[(\mathbf{e}_1 \times \mathbf{e}_2 ) \cdot \mathbf{e}_3]^2 &= [(\mathbf{e}_1 \times \mathbf{e}_2 ) \cdot \mathbf{e}_3][(\mathbf{e}_1 \times \mathbf{e}_2 ) \cdot \mathbf{e}_3]\\
            & = \left| \begin{array}{ccc}
        x_1 & y_1 & z_1  \\
        x_2 & y_2 & z_2  \\
        x_3 & y_3 & z_3
    \end{array}\right|\left| \begin{array}{ccc}
        x_1 & x_2 & x_3  \\
        y_1 & y_2 & y_3  \\
        z_1 & z_2 & z_3
    \end{array}\right|\\
        &= \left| \left(\begin{array}{ccc}
        x_1 & y_1 & z_1  \\
        x_2 & y_2 & z_2  \\
        x_3 & y_3 & z_3
    \end{array} \right) \left(\begin{array}{ccc}
        x_1 & x_2 & x_3  \\
        y_1 & y_2 & y_3  \\
        z_1 & z_2 & z_3
    \end{array}\right)\right|
    \end{aligned}
\end{equation}
由于
\begin{equation}
    \boldsymbol{e}_i\cdot \boldsymbol{e}_j = x_ix_j + y_iy_j + z_iz_j \quad i,j = 1,2,3
\end{equation}
所以
\begin{equation}
    \begin{aligned}
        \left(\begin{array}{ccc}
        x_1 & y_1 & z_1  \\
        x_2 & y_2 & z_2  \\
        x_3 & y_3 & z_3
    \end{array} \right) \left(\begin{array}{ccc}
        x_1 & x_2 & x_3  \\
        y_1 & y_2 & y_3  \\
        z_1 & z_2 & z_3
    \end{array}\right) &= \left( \begin{array}{ccc}
         \boldsymbol{e}_1 \cdot \boldsymbol{e}_1& \boldsymbol{e}_1 \cdot \boldsymbol{e}_2 & \boldsymbol{e}_1 \cdot \boldsymbol{e}_3 \\
         \boldsymbol{e}_2 \cdot \boldsymbol{e}_1& \boldsymbol{e}_2 \cdot \boldsymbol{e}_2 & \boldsymbol{e}_2 \cdot \boldsymbol{e}_3 \\
         \boldsymbol{e}_3 \cdot \boldsymbol{e}_1& \boldsymbol{e}_3 \cdot \boldsymbol{e}_2 & \boldsymbol{e}_3 \cdot \boldsymbol{e}_3 \\
    \end{array}\right) \\
   & = \left(\begin{array}{ccc}
        g_{11} & g_{12} & g_{13}  \\
         g_{21} & g_{22} & g_{23}  \\
         g_{31} & g_{32} & g_{33}  \\
    \end{array}\right) \\
    & = g
    \end{aligned}
\end{equation}
综上所述:
\begin{align}
    [(\boldsymbol{e}_1 \times \boldsymbol{e}_2 ) \cdot \boldsymbol{e}_3]^2 &= detg  \\
    \Omega := (\boldsymbol{e}_1 \times \boldsymbol{e}_2 ) \cdot \boldsymbol{e}_3 &= \sqrt{det g}
\end{align}
其中$\Omega$定义如上,是基底$\{\boldsymbol{e}_1,\boldsymbol{e}_2,\boldsymbol{e}_3\}$生成的平行六面体的体积。
\subsection{倒易基底}
由于$det g>0$,度量矩阵g有逆。记$g^{-1}$的元素为$g^{ij}$:
\begin{equation}
    g^{-1} = \left( \begin{array}{ccc}
         g^{11} & g^{12} & g^{13}  \\
         g^{21} & g^{22} & g^{23}  \\
         g^{31} & g^{32} & g^{33}  \\
    \end{array}\right)
\end{equation}
有逆矩阵的定义:
\begin{equation}
    g^{-1} g = gg^{-1} = I
\end{equation}
其中$I$是单位矩阵,有:
\begin{equation}
    g^{ij} g_{jk} = g_{ij} g^{jk} = \delta_{ik}
\end{equation}
其中:
\begin{equation}
    \delta_{ik} = \left\{ \begin{array}{rl}
         1, & \text{if } i=j  \\
         0, & \text{if } i\neq j
    \end{array}\right.
\end{equation}

\par 利用$g^{-1}$,我们定义倒易基矢$\boldsymbol{e}^i$:
\begin{equation}
    \boldsymbol{e}^{i} := g^{ij}\boldsymbol{e}_j\qquad i = 1,2,3.
\end{equation}
写成矩阵形式就是:
\begin{equation}
\begin{aligned}
        \left(
    \begin{array}{c}
         \boldsymbol{e}^1  \\
         \boldsymbol{e}^2  \\
         \boldsymbol{e}^3  \\
    \end{array}\right) &= g^{-1}  \left(
    \begin{array}{c}
         \boldsymbol{e}_1  \\
         \boldsymbol{e}_2  \\
         \boldsymbol{e}_3  \\
    \end{array}\right)\\
    & = \left( \begin{array}{ccc}
         g^{11} & g^{12} & g^{13}  \\
         g^{21} & g^{22} & g^{23}  \\
         g^{31} & g^{32} & g^{33}  \\
    \end{array}\right)\left(
    \begin{array}{c}
         \boldsymbol{e}_1  \\
         \boldsymbol{e}_2  \\
         \boldsymbol{e}_3  \\
    \end{array}\right)
\end{aligned}
\end{equation}

等式两边同时乘以$g$,有
\begin{equation}
    \begin{aligned}
     \left(
    \begin{array}{c}
         \boldsymbol{e}_1  \\
         \boldsymbol{e}_2  \\
         \boldsymbol{e}_3  \\
    \end{array}\right) = g\left(
    \begin{array}{c}
         \boldsymbol{e}^1  \\
         \boldsymbol{e}^2  \\
         \boldsymbol{e}^3  \\
    \end{array}\right) 
    \end{aligned}
\end{equation}
或者可以写成
\begin{equation}
    \boldsymbol{e}_i = g_{ij} \boldsymbol{e}^j
\end{equation}
我们称有倒易基矢构成的基底$\{\boldsymbol{e}^1,\boldsymbol{e}^2,\boldsymbol{e}^3\}$为倒易基底。
\par 现在我们来看一下倒易基矢的性质。首先是倒易基矢乘以正基矢的情况:
\begin{equation}
    \begin{aligned}
        \boldsymbol{e}^i \cdot \boldsymbol{e}_j = (g^{ik}\boldsymbol{e}_k)\cdot \boldsymbol{e}_j = g^{ik} (\boldsymbol{e}_k \cdot \boldsymbol{e}_j) = g^{ik}g_{kj} = \delta_{ij}
    \end{aligned}
\end{equation}
然后是倒易基矢之间的乘积:
\begin{equation}
\begin{aligned}
    \boldsymbol{e}^i\cdot \boldsymbol{e}^j &= (g^{ik}\boldsymbol{e}_k)\cdot (g^{jl}\boldsymbol{e}_l) = g^{ik}g^{jl}(\boldsymbol{e}_k \cdot \boldsymbol{e}_l) \\
    &= g^{ik}g^{jl}g_{kl} = g^{ik}g^{jl}g_{lk} =g6{ik}\delta_{jk} =g^{ij}
\end{aligned}
\end{equation}
\par 类似与正基底,倒易基底$\{\boldsymbol{e}^1,\boldsymbol{e}^2,\boldsymbol{e}^3\}$所生成的平行六面体的体积$\Omega^*$为:
\begin{equation}
    \Omega^* := (\boldsymbol{e}^1 \times \boldsymbol{e}^2)\cdot \boldsymbol{e}^3 = \sqrt{detg^-1} = \frac{1}{\sqrt{det g}} = \frac{1}{\Omega}
\end{equation}
所以$\Omega^*$和$\Omega$满足如下关系:
\begin{equation}
    \Omega^* \Omega = 1
\end{equation}
\par 下面我们再来具体讨论一下正、倒基矢的关系。由$\boldsymbol{e}^i \cdot \boldsymbol{e}_j = \delta_{ij}$
,有:
\begin{equation}
    \boldsymbol{e}^1\perp \boldsymbol{e}_2, \quad \boldsymbol{e}^1 \perp \boldsymbol{e}^3
\end{equation}
因此,
\begin{equation}
    \boldsymbol{e}^1 \parallel \boldsymbol{e}_2 \times \boldsymbol{e}_3
\end{equation}
于是,
\begin{equation}
    \boldsymbol{e}^1 = \lambda \boldsymbol{e}_2 \times \boldsymbol{e}_3
\end{equation}
利用$\boldsymbol{e}_1 \boldsymbol{e}^1 = $可以确定系数$\lambda$。这里我直接给出结果:
\begin{equation}
    \boldsymbol{e}^1 = \frac{1}{\Omega}\boldsymbol{e}_2 \times \boldsymbol{e}_3.
\end{equation}
同理
\begin{align}
    \boldsymbol{e}^2 = \frac{1}{\Omega}\boldsymbol{e}_3 \times \boldsymbol{e}_1,\\
    \boldsymbol{e}^3 = \frac{1}{\Omega}\boldsymbol{e}_1 \times \boldsymbol{e}_2
\end{align}
反之
\begin{align}
    \boldsymbol{e}_1 = \frac{1}{\Omega^*}\boldsymbol{e}^2 \times \boldsymbol{e}^3 ,\\
    \boldsymbol{e}_2 = \frac{1}{\Omega^*}\boldsymbol{e}^3 \times \boldsymbol{e}^1 ,\\
    \boldsymbol{e}_3 = \frac{1}{\Omega^*}\boldsymbol{e}^1 \times \boldsymbol{e}^2 ,
\end{align}
\subsection{逆变矢量与协变矢量}
有了正、倒基底,就可以搞点骚东西。对于两个矢量点积,我们可以将一个矢量用正基底展开,另外一个基底用倒基底展开
\begin{equation}
\begin{aligned}
        \mathbf{a}\cdot \mathbf{b} &= (a^i\mathbf{{e}_i})\cdot (b_j \mathbf{e}^j) = a^ib_j (\mathbf{e}_i \cdot \mathbf{e}^j) = a^i b_j \delta_{ij} = a^i b_i\\
    \mathbf{a}\cdot \mathbf{b} &= (a_i \mathbf{e}^i)\cdot (b^j\mathbf{e}_j) = a_i b^j(\mathbf{e}^i \cdot \mathbf{e}_j) = a_i b^j \delta_{ij} =a_i b^j
\end{aligned}
\end{equation}
我们习惯将第一种称为逆变形式,第二种称为协变形式。这样子我们可以把一个如下形式的求和:
\begin{equation}
    \sum_{i =1}^3 = x_i y_i = x_1y_1 + x_2 y_2 + x_3 y_3.
\end{equation}
看成欧式空间两个矢量$\mathbf{a}$和$\mathbf{b}$的点积,其中一个是正矢量,另一个是倒矢量。即
\begin{equation}
    \sum_{i = 1}^3x_iy_i = \mathbf{a}\cdot \mathbf{b},
\end{equation}
这里我们取
\begin{equation}
    a^i = x_i, \quad b_i =y_i,\qquad i = 1,2,3
\end{equation}
\section{正格矢与倒格矢}
假设我们为晶体建立了一个仿射坐标系$\{O; \mathbf{a}_1,\mathbf{a}_3,\mathbf{a}_3\}$。然后我们可以引入一个倒易基底$\{\mathbf{a}^1,\mathbf{a}^2,\mathbf{a}^3\}$。然而,在固体理论中,波矢与位矢的点乘很重要。这个点乘显然是一个位相,有自然的周期$2\pi$,因此为了方便期间,我们约定倒易基底$\{\mathbf{b}_1,\mathbf{b}_2,\mathbf{b}_3\}$如下,
\begin{equation}
    \mathbf{b}_i = 2\pi \mathbf{a}^i, \quad i = 1,2,3.
\end{equation}
根据上一节的内容,我们可以得到如下关系:
\begin{equation}
\begin{aligned}
        \mathbf{b}_1 = \frac{2\pi}{\Omega}\mathbf{a}_2 \times \mathbf{a}_3,\\
    \mathbf{b}_2 = \frac{2\pi}{\Omega}\mathbf{a}_3 \times \mathbf{a}_1,\\
    \mathbf{b}_3 = \frac{2\pi}{\Omega}\mathbf{a}_1 \times \mathbf{a}_2,
\end{aligned}
\end{equation}
其中,$\Omega$为正元胞的体积:
\begin{equation}
    \Omega = \mathbf{a}_1 \cdot (\mathbf{a}_2 \times \mathbf{a}_3).
\end{equation}
在此约定下,定义正倒格矢分别为:
\begin{align}
    \mathbf{R}_l = l_1 \mathbf{a}_1 + l_2 \mathbf{a}_2 + l_3 \mathbf{a}_3,\quad l_1,l_2,l_3 \in \mathbb{Z}  \\
    \mathbf{K}_n = n_1 \mathbf{b}_1 + n_2 \mathbf{b}_2 n_3 \mathbf{b}_3, \quad n_1,n_2,n_3 \in \mathbb{Z},
\end{align}
以及$\Omega$和$\Omega^*$满足:
\begin{equation}
    \Omega\Omega^* = (2\pi)^3
\end{equation}
其中,$\Omega^*$为倒元胞的体积
\begin{equation}
    \Omega^* = \mathbf{b}_1 \cdot (\mathbf{b}_2\times \mathbf{b}_3).
\end{equation}
然后就是这样规定下的正、倒基矢满足
\begin{equation}
    \mathbf{b}_i \cdot \mathbf{a}_j = 2\pi \delta_{ij}, \quad i,j = 1,2,3,
\end{equation}
利用这个关系可以得到
\begin{equation}
    \mathbf{K}_n \cdot \mathbf{R}_l = 2\pi\sum_{i =1 }^3n_i l_i = 2\pi m, \quad m \in \mathbb{Z}
\end{equation}
在物理上,人们一般将波矢放在倒空间,为倒矢量;将位矢安置在正空间,为正矢量。也就是说,上面的东西在物理上就写为:

\begin{align}
        \mathbf{k} = k_1 \mathbf{b}_1 + k_2 \mathbf{b}_2 + k_3 \mathbf{b}_3 \\
        \mathbf{r} = x_1 \mathbf{a}_1 + x_2 \mathbf{a}_2 + x_3 \mathbf{a}_3
\end{align}
于是有
\begin{equation}
    \mathbf{k}\cdot \mathbf{r} = \sum_{i = 1}^3 \sum_{j=1}^3 k_ix_j \mathbf{b}_i \cdot \mathbf{a}_j = \sum_{i = 1}^3 \sum_{j=1}^3k_ix_j 2\pi \delta_{ij} = 2\pi \sum_{i=1}^3 k_i x_i.
\end{equation}

\section{平移对称性}
假设我们有正格矢
\begin{equation}
    \mathbf{R}_l = l_1 \mathbf{a}_1 + l_2 \mathbf{a}_2 + l_3 \mathbf{a}_3,\quad l_1,l_2,l_3 \in \mathbb{Z} 
\end{equation}
为了方便,我们记平移操作$\mathbf{r}' = \mathbf{r} + \mathbf{R}_l$为
\begin{equation}
    \{E|\mathbf{R}_l\}
\end{equation}
其中记号中的$E$表示点变换的矩阵为单位矩阵。由所有这样的平移操作所构成的集合,我们记为$G_l$。
\begin{equation}
    G_l \equiv \{ \{E|\mathbf{R}_l\}| \mathbf{R}_l = l_1 \mathbf{a}_1 + l_2 \mathbf{a}_2 + l_3 \mathbf{a}_3,\quad l_1,l_2,l_3 \in \mathbb{Z}  \}
\end{equation}
可以验证,$G_l$是一个阿贝尔群。这里我就不证明了。
\par 现在我们来,考虑周期边界条件的影响。考虑如下周期边界条件:
\begin{align}
    \{E|\mathbf{R}_l\} \equiv \{E|\mathbf{R}_l + N_1 \mathbf{a}_1\},\\
    \{E|\mathbf{R}_l\} \equiv \{E|\mathbf{R}_l + N_2 \mathbf{a}_2\},\\
    \{E|\mathbf{R}_l\} \equiv \{E|\mathbf{R}_l + N_3 \mathbf{a}_3\},\\
\end{align}
在这样的周期边界条件下,系统的独立平移操作有只有$N = N_1N_2N_3$个。我们称所有独立操作构成的群为晶体平移群
\begin{equation}
    G_t = \{ \{E|\mathbf{R}_l\}| \mathbf{R}_l = l_1 \mathbf{a}_1 + l_2 \mathbf{a}_2 + l_3 \mathbf{a}_3,\quad 0\leq l_1<N_1,0\leq l_2 <N_2, 0\leq l_3<N_3  \}
\end{equation}
可以验证$G_t$仍然是一个阿贝尔群。(毕竟是$G_l$的子群嘛)
\subsection{晶体的平移对称性}
我们记平移变换操作为$\sigma$,设变换前函数为$f(p)$,变换后的函数为$g(p)$。
由之前学的群论知识,有
\begin{equation}
    g(p) = f(\sigma^{-1}(p)) 
\end{equation}
又有
\begin{equation}
    g =\sigma(f) = \sigma f
\end{equation}
联立上面两个等式有
\begin{equation}
\begin{aligned}
        (\sigma f) (p) = f(\sigma^{-1}(p))  \\
        \sigma f(p) = f(\sigma^{-1}p).
\end{aligned}
\end{equation}
等式左边的$\sigma$作用在函数上,而右边的$\sigma$作用在空间点上。具体怎么推我忘记了,但群论中有证明,一个是函数空间,另外一个是群空间。不管了,现在直接用。
\par 我们任取一个仿射坐标,设点$p$的坐标为$\mathbf{r}$,则
\begin{equation}
    \sigma f(\mathbf{r}) = f(\sigma^{-1} \mathbf{r})
\end{equation}
如果我们取$\sigma = \{E| \mathbf{R}_l \in G_t\}$,则
\begin{equation}
    \{E| \mathbf{R}_l\} f(\mathbf{r}) = f( \{E| \mathbf{R}_l\}^{-1} \mathbf{r})
\end{equation}
\par 由晶体平移对称性:
\begin{equation}
[g,H] = 0   \qquad \forall g \in G_t
\end{equation}

这个也可以证明,参见肖明文老师的笔记。
\section{Bloch定理}
我们记平移群的乘法为
\begin{equation}
    \{E| \mathbf{R}_l\} \{E| \mathbf{R}_{l'}\} = \{E| \mathbf{R}_l + \mathbf{R}_{l'}\}
\end{equation}
这样,平移群可以写为
\begin{equation}
    G_l = \{\{E|\mathbf{a}_1\}^{l_1}\{E|\mathbf{a}_2\}^{l_2}\{E|\mathbf{a}_3\}^{l_3} | l_1,l_2,l_3 \in \mathbb{Z}\}
\end{equation}

那么周期边界条件就记为
\begin{equation}
\begin{aligned}
        \{E| \mathbf{R}_l\} = \{E| \mathbf{R}_l\}\{E| \mathbf{a}_1\}^{N_1}\\
        \{E| \mathbf{R}_l\} = \{E| \mathbf{R}_l\}\{E| \mathbf{a}_2\}^{N_2}\\
        \{E| \mathbf{R}_l\} = \{E| \mathbf{R}_l\}\{E| \mathbf{a}_3\}^{N_3}\\
\end{aligned}
\end{equation}
现在我们考虑生成元
\begin{equation}
    \{E|\mathbf{a}_1\},\{E|\mathbf{a}_2\},\{E|\mathbf{a}_3\}
\end{equation}
由于平移群是阿贝尔群,以及$[g,H]=0$。所以你可以找到一个共同本征态$\psi(\mathbf{r})$满足
\begin{equation}
\begin{aligned}
        \{E|\mathbf{a}_1\} \psi(\mathbf{r}_) = \lambda_1 \psi(\mathbf{r}) \\
     \{E|\mathbf{a}_2\} \psi(\mathbf{r}_) = \lambda_2 \psi(\mathbf{r}) \\
      \{E|\mathbf{a}_3\} \psi(\mathbf{r}_) = \lambda_3 \psi(\mathbf{r}) \\
\end{aligned}
\end{equation}
其中$\lambda_1,\lambda_2,\lambda_3$是对应的本征值。那么对于平移群的任意一个元素,有
\begin{equation}
     \{E| \mathbf{R}_l\} = \{E|\mathbf{a}_1\}^{l_1}\{E|\mathbf{a}_2\}^{l_2}\{E|\mathbf{a}_3\}^{l_3} \psi(\mathbf{r}) = \lambda_1^{l_1}\lambda_2^{l_2}\lambda_3^{l_3}\psi(\mathbf{r})
\end{equation}
现在我们来考虑周期边界条件
\begin{equation}
\begin{aligned}
    \{E|\mathbf{a}_1\}^{N_1}\psi(\mathbf{r}) &= \{E|0\}\psi(\mathbf{r})\\
    \lambda_1^{N_1} \psi(\mathbf{r}) &= \psi(\mathbf{r})
\end{aligned}
\end{equation}
解得
\begin{equation}
    \lambda_1 = e^{2\pi \frac{n_1}{N_1}}\quad  0\le n_1 < N_1
\end{equation}
对于$\lambda_2,\lambda_3$同理可以求解。联立这三个生成元的结果,可以知道它们一起共有$N =N_1N_2N_3$个本征值。每个本征值对应一个本征子空间,故它们将系统的Hilbert空间分成N个子空间。最后我们得到
\begin{align}
    \{E|\mathbf{a}_1\}\psi_{(n_1,n_2,n_3)}(\mathbf{r}) = e^{-i2\pi 
    \frac{n_1}{N_1}}\psi_{(n_1,n_2,n_3)}(\mathbf{r}) \\
    \{E|\mathbf{a}_2\}\psi_{(n_1,n_2,n_3)}(\mathbf{r}) = e^{-i2\pi 
    \frac{n_2}{N_2}}\psi_{(n_1,n_2,n_3)}(\mathbf{r}) \\
    \{E|\mathbf{a}_3\}\psi_{(n_1,n_2,n_3)}(\mathbf{r}) = e^{-i2\pi 
    \frac{n_3}{N_3}}\psi_{(n_1,n_2,n_3)}(\mathbf{r}) \\
\end{align}

这里,波函数
\begin{equation}
    \psi_{(n_1,n_2,n_3)}(\mathbf{r})
\end{equation}
是本征子空间$(n_1,n_2,n_3)$中的任意一个态矢量。$(n_1,n_2,n_3)$是量子数。
\par 利用上述结果,任意一个平移算符作用倒本征态上,有
\begin{equation}
    \{E|\mathbf{R}_l\}\psi_{(n_1,n_2,n_3)}(\mathbf{r}) = exp\left(
    -i2\pi \sum_{\nu = 1}^3 \frac{n_\nu}{N_\nu}l_\nu\right)\psi_{(n_1,n_2,n_3)}(\mathbf{r})
    \label{1.1}
\end{equation}
如果我们记正格矢,倒格矢分别为
\begin{align}
    \mathbf{R}_l &= \sum_{\nu =1 }^3 l_\nu \mathbf{a}_\nu \\
    \mathbf{k} &= \sum_{\nu = 1}^3 \frac{n_{\nu}}{N_\nu}\mathbf{b}_\nu
\end{align}
那么有
\begin{equation}
    2\pi \sum_{\nu = 1}^3 \frac{n_\nu}{N_\nu}l_\nu = \mathbf{k}\cdot \mathbf{R}_l
\end{equation}
则式(\ref{1.1})就可以写成
\begin{equation}
    \{E|\mathbf{R}_l\}\psi_{(n_1,n_2,n_3)}(\mathbf{r}) = e^{-i\mathbf{k}\cdot \mathbf{R}_l}\psi_{(n_1,n_2,n_3)}(\mathbf{r})
\end{equation}
这样,平移算子的第$\mathbf{k}$个本征值便为
\begin{equation}
    e^{-i\mathbf{k}\cdot \mathbf{R}_l}
\end{equation}
而波函数
\begin{equation}
    \psi (\mathbf{r})
\end{equation}
则是第$\mathbf{k}$个本征子空间的任一态矢量。
\par 注意到
\begin{equation}
\begin{aligned}
        \{E|\mathrm{R}_l\}\psi_\mathbf{k}(\mathbf{r}) = \psi_\mathbf{k}(\mathbf{r} - \mathbf{R}_l)
\end{aligned}
\end{equation}
则有
\begin{equation}
    \begin{aligned}
         \psi_\mathbf{k} (\mathbf{k + \mathbf{R}_l}) = e^{i\mathbf{k}\cdot \mathbf{R}_l}\psi_\mathbf{k}(\mathbf{r}) \\
         \psi_\mathbf{k}(\mathbf{r})e^{-i\mathbf{k}\cdot \mathbf{r}} = \psi_\mathbf{k} (\mathbf{k + \mathbf{R}_l}) e^{-i\mathbf{k}\cdot (\mathbf{r}+ \mathbf{R}_l)}
    \end{aligned}
\end{equation}
这就是Bloch定理。
如果令
\begin{equation}
    u_\mathbf{k}(\mathbf{r}) = \psi_\mathbf{k}(\mathbf{r}) e^{-i\mathbf{k}\cdot \mathbf{r}}
\end{equation}
则
\begin{equation}
    u_\mathbf{k}(\mathbf{r}) = u_\mathbf{k}(\mathbf{r}+ \mathbf{R}_l)
\end{equation}
\par 从晶体的倒空间来看,平移算子群$G_t$的全部量子数$\mathbf{k}$只占据以原点为中心的一个平行六面体
\begin{equation}
    \mathbf{k} = \sum_{\nu = 1}^3 \frac{n_\nu}{N_\nu}\mathbf{b}_\nu,\quad -\left[\frac{N_\nu}{2}\right] \le n_\nu < \left[\frac{N_\nu}{2}\right]
\end{equation}
通常称该平行六面体为第一布里渊区简称$BZ$。
BZ之外的任意一个波矢$\mathbf{k}'$总可以写成
\begin{equation}
    \mathbf{k}' =\mathbf{k} + \mathbf{K}_n
\end{equation}
其中,$\mathbf{k} \in BZ$,而$\mathbf{K}_n$则是某一倒格矢
\begin{equation}
    \mathbf{K}_n = \sum_\nu n_\nu b \quad n_\nu in \mathbb{Z}
\end{equation}
\par 值得指出,对于平移群的全部本征值构成的集合
\begin{equation}
    \{e^{i\mathbf{k}\cdot \mathbf{R}_l}| \mathbf{k} \in BZ, 0 \le l_\nu < N_\nu, \nu = 1,2,3\}
\end{equation}
有两条性质
\begin{align}
    \frac{1}{N}\sum_{\mathbf{R}_l} e^{i(\mathbf{k}- \mathbf{k'})\cdot \mathbf{R}_l} &= \delta_{\mathbf{kk}'},\\
    \frac{1}{N}\sum_{\mathbf{R}_l} e^{i\mathbf{k}\cdot (\mathbf{R}_l - \mathbf{R}_l')} &= \delta_{\mathbf{R}_l\mathbf{R}_{l'}},
\end{align}
\subsection{求本征方程}
\par 对于第$\mathbf{k}$个子空间,波矢
\begin{equation}
    \psi_\mathbf{k} (\mathbf{r}) = u_\mathbf{k}(\mathbf{r}) e^{i\mathbf{k}\cdot \mathbf{r}}
\end{equation}
本征方程
\begin{equation}
    H\psi_\mathbf{k} (\mathbf{r}) = E(\mathbf{k}) \psi_\mathbf{k} (\mathbf{r})
\end{equation}
哈密顿算符
\begin{equation}
    H =- \frac{\hbar^2}{2m}\nabla^2 + U(\mathbf{r})
\end{equation}
利用矢量分析公式可得
\begin{equation}
    [- \frac{\hbar^2}{2m}\nabla^2  - \frac{i\hbar^2}{m}\mathbf{k}\cdot \nabla + \frac{\hbar^2 \mathbf{k}^2}{2m}+ U(\mathbf{r})]u_\mathbf{k}(\mathbf{r}) =E(\mathbf{k}) u_\mathbf{k}(\mathbf{r})  
\end{equation}
如果令
\begin{equation}
\begin{aligned}
        H_\mathbf{k}\equiv- \frac{\hbar^2}{2m}\nabla^2  - \frac{i\hbar^2}{m}\mathbf{k}\cdot \nabla + \frac{\hbar^2 \mathbf{k}^2}{2m}+ U(\mathbf{r}) &= H - \frac{\hbar^2}{2m}\nabla^2  - \frac{i\hbar^2}{m}\mathbf{k}\cdot \nabla \\
        &= \frac{1}{2m}(\mathbf{p} + \hbar \mathbf{k})^2 + U(\mathbf{r})
\end{aligned}
\end{equation}

则本征方程可以写为
\begin{equation}
    H_\mathbf{k}u_\mathbf{k}(\mathbf{r}) = E(\mathbf{k}) u_\mathbf{k}(\mathbf{r})
\end{equation}
其中$u_\mathbf{k}(\mathbf{r})$满足周期边界条件
\begin{equation}
    u_\mathbf{k}(\mathbf{r}) = u_\mathbf{k}(\mathbf{r} + \mathbf{R}_l)
\end{equation}
\section{Dirac Comb}
现在我们来讨论一个简单的能带模型--Dirac Comb,其周期势为
\begin{equation}
    U(x) = \gamma \sum_{l = -\infty}^{+\infty}\delta(x + la), \quad \gamma >0,a>0,l\in \mathbf{Z},
\end{equation}
在区间$(-a,0)$的本征方程为
\begin{equation}
(-\frac{\hbar^2}{2m} \frac{d^2}{dx^2} - \frac{i\hbar^2}{m}k \frac{d}{dx} + \frac{\hbar^2k^2}{2m}) u_k(x) = E(k)u_k(x) \quad x\in (-a,0)
\end{equation}
令$q = \sqrt{\frac{2m}{\hbar^2}E(k)}$,经过化简后方程变为
\begin{equation}
    (-\frac{d^2}{dx^2} - 2ik\frac{d}{dx} + k^2 - q^2)u_k(x) = 0
\end{equation}
这是一个常系数线性齐次微分方程,利用大学的微积分知识解得
\begin{equation}
    u_k(x) = Ae^{i(q-k)x} + Be^{-i(q_k)x}, \quad x\in (-a,0)
\end{equation}
利用周期边界条件$u_k(x+la) = u_k(x)$,我们可以得到在$(0,a)$的解
\begin{equation}
    u_k(x) = u_k(x-a) = Ae^{i(q-k)(x-a)} + Be^{-i(q_k)(x-a)}, \quad x\in (0,a)
\end{equation}

现在我们考虑交接点$x = 0$来确定待定系数。首先,函数$u_k(x)$在交接点连续
\begin{equation}
    u_k(x)|_{x=0-} = u_k(x)|_{x=0+}
\end{equation}
其次,对$(-a,a)$的本征方程
\begin{equation}
    [-\frac{\hbar^2}{2m} \frac{d^2}{dx^2} - \frac{i\hbar^2}{m}k \frac{d}{dx} + \frac{\hbar^2k^2}{2m} + \gamma \delta(x)] u_k(x) = E(k)u_k(x)
\end{equation}
积分后可以得到
\begin{equation}
    u_k'(0+0 -u_k'(0-) = \frac{2m\gamma}{\hbar^2}u_k(0).
\end{equation}
\par 利用这两个边界条件后可以得到如下方程组
\begin{align}
    A + B &= Ae^{-i(q-k)a} + Be^{i(q+k)a} \\
    i(q-k)Ae^{-i(q-k)a}&- i (q+k)Be^{i(q+k)a} - i(q-k)A + i(q+k)B = \frac{2m\gamma}{\hbar^2}(A+B)
\end{align}
后面的工作就是解这个方程组了。化简后得到
\begin{align}
    A + B = e^{ika}(A e^{-iqa} + Be^{iqa})  \\
    (1 - 2\frac{i\kappa}{q})A - (1+ 2\frac{i\kappa}{q}) B = e^{ika}(Ae^{-iqa} - Be^{iqa})
\end{align}
其中$\kappa \equiv \frac{m\gamma}{\hbar}$。再进一步化简,可以写成如下的矩阵形式
\begin{equation}
    \left(\begin{array}{cc}
        (1 + \frac{i\kappa}{q}) e^{-iqa} & \frac{i\kappa}{q}e^{iqa} \\
        -\frac{i\kappa}{q} e^{-iqa} & (1 - \frac{i\kappa}{q})e^{iqa}
    \end{array}\right)\left(\begin{array}{c}
         A  \\
         B 
    \end{array}\right) = e^{-ika}\left(\begin{array}{c}
         A  \\
         B 
    \end{array}\right)
\end{equation}
这就要考虑一个问题了。对于上面的这样一个矩阵,要考虑有解的条件,即
\begin{equation}
     \left|\begin{array}{cc}
        (1 + \frac{i\kappa}{q}) e^{-iqa}-e^{-ika} & \frac{i\kappa}{q}e^{iqa} \\
        -\frac{i\kappa}{q} e^{-iqa} & (1 - \frac{i\kappa}{q})e^{iqa}-e^{-ika}
    \end{array}\right| = 0
\end{equation}
展开并化简后得到
\begin{equation}
    \begin{aligned}
        cos(qa) + \frac{\kappa}{q}sin(qa) = cos(ka), \quad k\in BZ.
    \end{aligned}
    \label{1.2}
\end{equation}
设其第$n$个解为$q_n$,则由$q$的定义$q = \sqrt{\frac{2m}{\hbar^2}E(k)}$可以得到电子的第n个能级为
\begin{equation}
    E_n(k) = \frac{\hbar^2q_n^2}{2m}
\end{equation}
式(\ref{1.2})是一个关于$q$的超越方程,一般只能数值求解。