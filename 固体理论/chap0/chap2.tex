\chapter{声子}
第二章就是声子了,感觉内容很多,数学运算也很多。感觉对于这样的内容如果只是简单的几个笔记,学习一遍,那我认为肯定是不会学的很懂。但如果不这么做,我肯定更不懂。我想表达的是,慢慢来,别太急,
我觉得你现在开始整理笔记,并努力在执行费曼学习法,这点已经超越之前的自己很多了。但考虑到学习过程中有许多问题可能一下没有转过来,或者有些疑问,我想要记录下来,如果以后有一天我还记得现在
做的笔记,并且有耐心慢慢的读,而且有时间的话,我希望未来的我能够解决这些问题。也不一定是多久以后,也许明天,也许下周?
\section{声场量子化}
这一节的目的就是一步到位把声场给量子化了。不多废话。
\subsection{模型}
我们讨论简单晶格,每个元胞只有一个原子或离子,它在平衡位置附近做微小震荡
\begin{equation}
    \begin{aligned} 
    \mathbf{X}_\mathbf{l} =\mathbf{R_l}  + \mathbf{u_l}(t)  
    \end{aligned} 
\end{equation}
其中, $ \mathbf{R_l}   $ 是平衡位置, 而 $ \mathbf{u_l} $ 是对平衡位置的微小偏离。
\begin{equation}
    \begin{aligned} 
    \mathbf{R_l} &= \sum_{\nu = 1}^3 l_\nu \mathbf{a}_\nu \\
    \mathbf{u_l} &= \sum_{\nu = 1}^3 u_\mathbf{l}^\nu   \mathbf{a}_\nu 
    \end{aligned} 
\end{equation}
所以说
在小振动问题上,做简谐近似,即将系统的势能 $ \mathbf{\Psi}  $ 对小偏离作 Taylor 展开,并且只
保留到二届项:
\begin{equation}
    \begin{aligned} 
    \Phi = \Phi_0 + \sum_{\mathbf{l},\alpha }\Phi_\alpha (\mathbf{l}) u_\mathbf{l}^\alpha 
    + \frac{1}{2} \sum_{\mathbf{l},\alpha }\sum_{\mathbf{l'},\beta}\Phi_{\alpha \beta} 
    (\mathbf{l,l'} )u_\mathbf{l}^\alpha u_\mathbf{l'}^\beta 
    \end{aligned} 
\end{equation}
一方面,可以通过选取能量零点,使 $ \Phi_0 =0 $ ; 令一方面,在平衡位置晶体势能取极小值, $ \Phi_\alpha 
(\mathbf{l})  = \frac{\partial \Phi}{\partial u_\mathbf{l}^\alpha }|_0 = 0 $ 。此外,考虑晶体
的平移对称性
\begin{equation}
    \begin{aligned} 
    \Phi_{\alpha \beta} (\mathbf{l,l'} ) = \Phi_{\alpha \beta}(\mathbf{l - l'} ) = 
    \Phi_{\beta \alpha} (\mathbf{l' - l} ) = \Phi_{\beta \alpha}(\mathbf{l',l} )
    \end{aligned} 
\end{equation}
最后,在简谐近似条件下,晶格振动系统的势能化简为
\begin{equation}
    \begin{aligned} 
        \Phi = \frac{1}{2} \sum_{\mathbf{l},\alpha }\sum_{\mathbf{l'},\beta}\Phi_{\alpha \beta} 
        (\mathbf{l-l'} )u_\mathbf{l}^\alpha u_\mathbf{l'}^\beta 
    \end{aligned} 
\end{equation}
\par 于是,系统的哈密顿量为
\begin{equation}
    \begin{aligned} 
    H = \frac{1}{2m}\sum_{\mathbf{l},\alpha}p_\mathbf{l}^\alpha  p_\mathbf{l}^\alpha   + 
    \frac{1}{2}\sum_{\mathbf{l},\alpha }\sum_{\mathbf{l'},\beta } \Phi_{\alpha \beta} 
    (\mathbf{l-l'} )u_\mathbf{l}^\alpha u_\mathbf{l'}^\beta 
    \end{aligned} 
\end{equation}
为了处理方便,做如下变换
\begin{equation}
    \begin{aligned} 
    u_\mathbf{l}^\alpha \rightarrow \tilde{u}_\mathbf{l}^\alpha = \sqrt{m} u_\mathbf{l}^\alpha \\
    p_\mathbf{l}^\alpha \rightarrow \tilde{p}_\mathbf{l}^\alpha = \sqrt{m} p_\mathbf{l}^\alpha
    \end{aligned} 
\end{equation}
变换后哈密顿量变为
\begin{equation}
    \begin{aligned} 
        H = \frac{1}{2}\sum_{\mathbf{l},\alpha}p_\mathbf{l}^\alpha  p_\mathbf{l}^\alpha   + 
    \frac{1}{2m}\sum_{\mathbf{l},\alpha }\sum_{\mathbf{l'},\beta } \Phi_{\alpha \beta} 
    (\mathbf{l-l'} )u_\mathbf{l}^\alpha u_\mathbf{l'}^\beta 
    \label{chap1-1}
    \end{aligned} 
\end{equation}
这里值得一提的是,对 $ \mathbf{l}  $ 求和表示对于晶格的所有格点求和。对 $ \alpha \in {1,2,3} $ 求和则是
对正则坐标的三个方向求和。这样总的求和就有 $ 3N $ 个了。
\subsection{傅里叶变换}
也许是方便求和吧,所以要作傅里叶变换。因为作了变换后能算出个结果来。傅里叶变换如下
\begin{equation}
    \begin{aligned} 
    u_\mathbf{l}^\alpha = \frac{1}{\sqrt{N}} \sum_\mathbf{k} Q_{\mathbf{k}\alpha} e^{i \mathbf{k\cdot l}} \\
    p_\mathbf{l}^\alpha = \frac{1}{\sqrt{N}} \sum_\mathbf{k} P_{\mathbf{k}\alpha} e^{i \mathbf{k\cdot l}} \\ 
    \end{aligned} 
\end{equation}
由傅里叶变换的数学知识,系数表达式如下
\begin{equation}
    \begin{aligned} 
    Q_{\mathbf{k} \alpha } = \frac{1}{\sqrt{N}} \sum_\mathbf{l} u_\mathbf{l}^\alpha e^{-i \mathbf{k\cdot l} } \\  
    P_{\mathbf{k} \alpha } = \frac{1}{\sqrt{N}} \sum_\mathbf{l} p_\mathbf{l}^\alpha e^{i \mathbf{k\cdot l} } 
    \end{aligned} 
\end{equation}
\par 在量子力学中 $u_\mathbf{l}^alpha, p_\mathbf{l}^\alpha$都是厄米算符(textcolor{red}{这里由于
我比较懒,和参考资料的原因,没有打上标,就理解成算符吧}),满足
\begin{equation}
    \begin{aligned} 
    (u_\mathbf{l}^\alpha )^\dagger = u_\mathbf{l}^\alpha \\
    (p_\mathbf{l}^\alpha )^\dagger = p_\mathbf{l}^\alpha
    \end{aligned} 
\end{equation}  
再根据 $Q_{\mathbf{k}\alpha },P_{\mathbf{k} \alpha }$的定义,两边取共轭,就可以得到新的坐标和动量
满足
\begin{equation}
    \begin{aligned} 
    Q_{\mathbf{k}\alpha }^\dagger = Q_{-\mathbf{k}alpha } \\
    P_{\mathbf{k}\alpha }^\dagger = P_{-\mathbf{k}alpha }
    \end{aligned} 
\end{equation}
在 Fourier 变换下,哈密顿量的动能项
\begin{equation}
    \begin{aligned} 
    T &= \frac{1}{2}\sum_{\mathbf{l},\alpha}p_\mathbf{l}^\alpha  p_\mathbf{l}^\alpha  \\
      &= \frac{1}{2}\sum_\mathbf{k}\sum_\alpha  P_{\mathbf{k}\alpha }^\dagger P_{\mathbf{k}\alpha }
    \end{aligned} 
\end{equation}
具体计算从略,如果不懂,可以再推一遍,这里面有个求和我认为值得注意一下
\begin{equation}
    \begin{aligned} 
    \frac{1}{N} \sum_\mathbf{l} e^{-i(\mathbf{k} + \mathbf{k'})\cdot \mathbf{l}} = \delta_{\mathbf{k',-k} } 
    \end{aligned} 
\end{equation}
\par 接下来是势能项
\begin{equation}
    \begin{aligned} 
    \Phi &= \frac{1}{2} \sum_{\mathbf{l},\alpha }\sum_{\mathbf{l'},\beta}\Phi_{\alpha \beta} 
    (\mathbf{l,l'} )u_\mathbf{l}^\alpha u_\mathbf{l'}^\beta \\
    &= \frac{1}{2m} \sum_\mathbf{k}\sum_{\alpha,\beta} Q_{\mathbf{k}\alpha }Q_{\mathbf{-k}\beta }
    \sum_{\mathbf{l} }\Phi_{\alpha \beta} e^{i \mathbf{k\cdot l} }\\
    &= \frac{1}{2} \sum_\mathbf{k}\sum_{\alpha,\beta} Q_{\mathbf{k}\alpha }Q_{\mathbf{-k}\beta }
    D_{\alpha \beta} (-\mathbf{k} ) \\
    &= \frac{1}{2}\sum_\mathbf{k}\sum_{\alpha,\beta} D_{\alpha \beta} (\mathbf{k} ) Q_{\mathbf{k}\alpha }^\dagger
    Q_{\mathbf{k}\alpha }
    \end{aligned} 
\end{equation}
其中 
\begin{equation}
    \begin{aligned} 
        D_{\alpha \beta} (\mathbf{k} ) = \frac{1}{m}\Phi_{\alpha \beta} (\mathbf{l} )e^{-i \mathbf{k\cdot l} }
    \end{aligned} 
\end{equation}
这样子,我们就得到了动量空间的哈密顿量表达式了
\begin{equation}
    \begin{aligned} 
    H = \frac{1}{2}\sum_\mathbf{k}\left[\sum_\alpha  P_{\mathbf{k}\alpha }^\dagger P_{\mathbf{k}\alpha } +  \sum_{\alpha,\beta} D_{\alpha \beta} (\mathbf{k} ) Q_{\mathbf{k}\alpha }^\dagger
    Q_{\mathbf{k}\alpha }\right] 
    \end{aligned} 
\end{equation}
其中,
\begin{equation}
    \begin{aligned} 
        P_{\mathbf{k}\alpha }^\dagger  &= P_{\mathbf{-k}\alpha } \\
        Q_{\mathbf{k}\alpha }^\dagger  &= Q_{-\mathbf{k}\alpha } \\
        D_{\alpha \beta} (\mathbf{k} ) &= \frac{1}{m}\Phi_{\alpha \beta} (\mathbf{l} )e^{-i \mathbf{k\cdot l} }
    \end{aligned} 
\end{equation}
\subsection{幺正变换}
系统哈密顿量
\begin{equation}
    \begin{aligned} 
    H &= \frac{1}{2}\sum_\mathbf{k}\left[\sum_\alpha  P_{\mathbf{k}\alpha }^\dagger P_{\mathbf{k}\alpha } +  \sum_{\alpha,\beta} D_{\alpha \beta} (\mathbf{k} ) Q_{\mathbf{k}\alpha }^\dagger
    Q_{\mathbf{k}\alpha }\right] 
     &= \frac{1}{2} \sum_\mathbf{k} h_\mathbf{k} 
    \end{aligned} 
\end{equation}
其中
\begin{equation}
    \begin{aligned} 
    h_\mathbf{k}  &= \sum_\alpha  P_{\mathbf{k}\alpha }^\dagger P_{\mathbf{k}\alpha } +  \sum_{\alpha,\beta} D_{\alpha \beta} (\mathbf{k} ) Q_{\mathbf{k}\alpha }^\dagger
    Q_{\mathbf{k}\alpha } \\
    &= \left(\begin{array}{ccc}
          P_{\mathbf{k}1}^\dagger  & P_{\mathbf{k}2}^\dagger  &P_{\mathbf{k}3}^\dagger 
    \end{array}\right) \left(\begin{array}{c}
        P_{\mathbf{k}1} \\
        P_{\mathbf{k}2} \\
        P_{\mathbf{k}3}
          \end{array}\right)+ \left(\begin{array}{ccc}
            Q_{\mathbf{k}1}^\dagger  & Q_{\mathbf{k}2}^\dagger  &Q_{\mathbf{k}3}^\dagger 
      \end{array}\right) D(\mathbf{k} )\left(\begin{array}{c}
          Q_{\mathbf{k}1} \\
          Q_{\mathbf{k}2} \\
          Q_{\mathbf{k}3}
            \end{array}\right)
    \end{aligned}
\end{equation}
插入单位元 $ I = U(\mathbf{k})U^\dagger (\mathbf{k}) $,其中幺正算符可以使动力学矩阵$D(\mathbf{k} )$变成
对角矩阵的形式,即
\begin{equation}
    \begin{aligned} 
        U^\dagger (\mathbf{k})D(\mathbf{k} )U(\mathbf{k}) = \left(\begin{array}{ccc}
            \lambda_{\mathbf{k}1 } &  0 & 0 \\
            0 & \lambda_{\mathbf{k}2 }  & 0 \\
            0 & 0 & \lambda_{\mathbf{k}3 } 
              \end{array}\right)
    \end{aligned} 
\end{equation} 
同时,我们记
\begin{equation}
    \begin{aligned} 
    \left(\begin{array}{c}
         q_{\mathbf{k}1 } \\
         q_{\mathbf{k}2 } \\
         q_{\mathbf{k}3 } 
         \end{array}\right) = U^\dagger (\mathbf{k} ) \left(\begin{array}{c}
            Q_{\mathbf{k}1 } \\
            Q_{\mathbf{k}2 } \\
            Q_{\mathbf{k}3 } 
            \end{array}\right) \\
            \left(\begin{array}{c}
                p_{\mathbf{k}1 } \\
                p_{\mathbf{k}2 } \\
                p_{\mathbf{k}3 } 
                \end{array}\right) = U^\dagger (\mathbf{k} ) \left(\begin{array}{c}
                   P_{\mathbf{k}1 } \\
                   P_{\mathbf{k}2 } \\
                   P_{\mathbf{k}3 } 
                   \end{array}\right)
    \end{aligned}  
\end{equation}
经过上述幺正变换后,就有
\begin{equation}
    \begin{aligned} 
    h_\mathbf{k} &=  \left(\begin{array}{ccc}
        p_{\mathbf{k}1}^\dagger  & p_{\mathbf{k}2}^\dagger  &p_{\mathbf{k}3}^\dagger 
  \end{array}\right) \left(\begin{array}{c}
      p_{\mathbf{k}1} \\
      p_{\mathbf{k}2} \\
      p_{\mathbf{k}3}
        \end{array}\right)+ \left(\begin{array}{ccc}
          q_{\mathbf{k}1}^\dagger  & q_{\mathbf{k}2}^\dagger  &q_{\mathbf{k}3}^\dagger 
    \end{array}\right) \left(\begin{array}{ccc}
        \lambda_{\mathbf{k}1 } &  0 & 0 \\
        0 & \lambda_{\mathbf{k}2 }  & 0 \\
        0 & 0 & \lambda_{\mathbf{k}3 } 
          \end{array}\right)
    \left(\begin{array}{c}
        q_{\mathbf{k}1} \\
        q_{\mathbf{k}2} \\
        q_{\mathbf{k}3}
          \end{array}\right) \\
          &= \sum_{\sigma = 1}^3 p^\dagger_{\mathbf{k}\sigma }p_{\mathbf{k}\sigma }
           + \sum_{\sigma = 1}^3 \lambda_{\mathbf{k}\sigma }q_{\mathbf{k}\sigma }^\dagger q_{\mathbf{k}\sigma }
    \end{aligned} 
\end{equation}
因此,系统的哈密顿量就化为
\begin{equation}
    \begin{aligned} 
    H &= \frac{1}{2} \sum_\mathbf{k} h_\mathbf{k} \\ 
    &= \frac{1}{2}\sum_{\mathbf{k}\sigma } p^\dagger_{\mathbf{k}\sigma }p_{\mathbf{k}\sigma }
    +\frac{1}{2} \sum_{\mathbf{k}\sigma} \lambda_{\mathbf{k}\sigma }q_{\mathbf{k}\sigma }^\dagger q_{\mathbf{k}\sigma }
    \end{aligned} 
\end{equation}
其中,
\begin{equation}
    \begin{aligned} 
    \lambda_{\mathbf{k}\sigma } \ge 0, \quad \sigma = 1,2,3
    \end{aligned} 
\end{equation}
可以验证,但我现在懒得验证。经过前面这么一大串变换后得到的坐标和动量算符仍然满足坐标与动量的对易关系,即
\begin{equation}
    \begin{aligned} 
    \relax [q_{\mathbf{k}\sigma },q_{\mathbf{k'}\sigma' }] = [p_{\mathbf{k}\sigma },p_{\mathbf{k'}\sigma' }] = 0
    \quad [q_{\mathbf{k}\sigma },p_{\mathbf{k'}\sigma }] =\delta_{\mathbf{kk'} }\delta_{\sigma \sigma '}
    \end{aligned} 
\end{equation}
\subsection{Dirac变换}
现在,我们来求解运动方程.
\par 由之前的对易关系和哈密顿量表达式,可以得到
\begin{equation}
    \begin{aligned} 
    \relax [q_{\mathbf{k}\sigma, H }] &= \frac{1}{2} \sum_{\mathbf{k'}\sigma' }[q_{\mathbf{k}\sigma}
    ,p^\dagger_{\mathbf{k}\sigma}p_{\mathbf{k}\sigma}] \\
    &= q_{\mathbf{-k}\sigma} 
    \end{aligned} 
\end{equation}
以及
\begin{equation}
    \begin{aligned} 
        \relax [p_{\mathbf{k}\sigma, H }] &=  \frac{1}{2} \sum_{\mathbf{k'}\sigma' } \lambda_{\mathbf{k'}\sigma' }
        [p_{\mathbf{k}\sigma},q^\dagger_{\mathbf{k'}\sigma'}q_{\mathbf{k'}\sigma'}]\\
        &= -\lambda_{\mathbf{k}\sigma }q_{\mathbf{k}\sigma} 
    \end{aligned} 
\end{equation}
其中这里应该是用了 $ \lambda_{\mathbf{-k}\sigma} = \lambda_{\mathbf{k}\sigma} $。将上面的计算结果
带入海森堡运动方程,有
\begin{equation}
    \begin{aligned} 
    \frac{d}{dt} q_{\mathbf{k}\sigma} = p_{\mathbf{k}\sigma} \\
    \frac{d}{dt} p_{\mathbf{k}\sigma} = -\lambda_{\mathbf{k}\sigma }q_{\mathbf{k}\sigma} 
    \end{aligned} 
\end{equation}
其中令 $ \frac{1}{i\hbar} = 1 $ 。联立上面两式得到
\begin{equation}
    \begin{aligned} 
    \ddot{q}_{\mathbf{k}\sigma } + \lambda_{\mathbf{k}\sigma }q_{\mathbf{k}\sigma}  = 0
    \end{aligned} 
\end{equation}
解得
\begin{equation}
    \begin{aligned} 
        q_{\mathbf{k}\sigma}(t) = a_{\mathbf{k}\sigma}e^{-i\omega_\sigma (\mathbf{k} )t}
        + b_{\mathbf{k}\sigma}e^{i\omega_\sigma (\mathbf{k} )t}
    \end{aligned} 
\end{equation}
其中 
\begin{equation}
    \begin{aligned} 
        \omega_\sigma (\mathbf{k} ) = \sqrt{\lambda_{\mathbf{k}\sigma }}
    \end{aligned} 
\end{equation}
注意到 $  q^\dagger_{\mathbf{k}\sigma}(t) = q_{\mathbf{-k}\sigma}(t)$,于是联立后有
\begin{equation}
    \begin{aligned} 
        a^\dagger_{\mathbf{k}\sigma}e^{i\omega_\sigma (\mathbf{k} )t}
        + b^\dagger_{\mathbf{k}\sigma}e^{-i\omega_\sigma (\mathbf{k} )t} =
        a_{\mathbf{k}\sigma}e^{-i\omega_\sigma (\mathbf{k} )t}
        + b_{\mathbf{k}\sigma}e^{i\omega_\sigma (\mathbf{k} )t}
    \end{aligned} 
\end{equation}
分别令 $ t = 0 $ 和 $\omega_\sigma (\mathbf{k} )t = \frac{\pi}{2}  $  ,得 
\begin{equation}
    \begin{aligned} 
        a^\dagger_{\mathbf{k}\sigma}
        + b^\dagger_{\mathbf{k}\sigma}=
        a_{\mathbf{k}\sigma}
        + b_{\mathbf{k}\sigma} \\
        a^\dagger_{\mathbf{k}\sigma}
        - b^\dagger_{\mathbf{k}\sigma}=
        -a_{\mathbf{k}\sigma}
        + b_{\mathbf{k}\sigma}
    \end{aligned} 
\end{equation}
解得 
\begin{equation}
    \begin{aligned} 
        b_{\mathbf{k}\sigma}  = a_{-\mathbf{k}\sigma}
    \end{aligned} 
\end{equation}
于是$q_{\mathbf{k}\sigma}(t)$的表达式化简为
\begin{equation}
    \begin{aligned} 
        q_{\mathbf{k}\sigma}(t) &= a_{\mathbf{k}\sigma}e^{-i\omega_\sigma (\mathbf{k} )t}
        + a^\dagger_{-\mathbf{k}\sigma}e^{i\omega_\sigma (\mathbf{k} )t} \\
            &= a_{\mathbf{k}\sigma} (t) + a^\dagger_{\mathbf{k}\sigma }(t)
    \end{aligned} 
\end{equation}
其中$a_{\mathbf{k}\sigma} (t) = a_{\mathbf{k}\sigma}e^{-i\omega_\sigma (\mathbf{k} )t}$。
同理$ p_{\mathbf{k}\sigma}(t)$的表达式
\begin{equation}
    \begin{aligned} 
        p_{\mathbf{k}\sigma}(t) &= \frac{d}{dt} q_{\mathbf{k}\sigma}
        &= -i\omega_{\sigma}(\mathbf{k} )[a_{\mathbf{k}\sigma} (t) - a^\dagger_{\mathbf{k}\sigma }(t)]
    \end{aligned} 
\end{equation}
联立上面两式,我们得到
\begin{equation}
    \begin{aligned} 
        a_{\mathbf{k}\sigma} (t) = \frac{1}{2}[ q_{\mathbf{k}\sigma}(t) + i\frac{1}{\omega_\sigma (\mathbf{k} )}
        p_{\mathbf{k}\sigma}(t)] \\ 
        a^\dagger_{\mathbf{k}\sigma} (t) = \frac{1}{2}[ q_{\mathbf{k}\sigma}(t) - i\frac{1}{\omega_\sigma (\mathbf{k} )}
        p_{\mathbf{k}\sigma}(t)] 
    \end{aligned} 
\end{equation}
由于分母出现了$\omega_\sigma (\mathbf{k} )$,以上变换只对 $ \omega_{\mathbf{k}\sigma } = \sqrt{\lambda_{\mathbf{k}\sigma }} >0 $
的情况适用。因此将系统哈密顿量分为两部分
\begin{equation}
    \begin{aligned} 
    H &= H_0 + H_1 \\
     &= \sum_{\mathbf{k}_0,\sigma }p^\dagger_{\mathbf{k}_0\sigma }p_{\mathbf{k}_0\sigma } 
     + \frac{1}{2}\sum_{\mathbf{k,\sigma }'} (\lambda_{\mathbf{k}\sigma }q_{\mathbf{k}\sigma }^\dagger q_{\mathbf{k}\sigma })
    \end{aligned} 
\end{equation} 
将之前的变换带入 $ H_1 $,得到
\begin{equation}
    \begin{aligned} 
    H_1 &= \frac{1}{2}\sum_{\mathbf{k},\sigma }' (p^\dagger_{\mathbf{k}\sigma }p_{\mathbf{k}\sigma }+\lambda_{\mathbf{k}\sigma }q_{\mathbf{k}\sigma }^\dagger q_{\mathbf{k}\sigma })\\
        &= \frac{1}{2}\sum_{\mathbf{k},\sigma}' [ - \omega_\sigma^2 (\mathbf{k})(a_{\mathbf{k}\sigma} - a^\dagger_{\mathbf{k}\sigma })(a_{\mathbf{k}\sigma} - a^\dagger_{\mathbf{k}\sigma })
        + \lambda_{\mathbf{k}\sigma } (a_{\mathbf{k}\sigma}  + a^\dagger_{\mathbf{k}\sigma })(a_{\mathbf{k}\sigma}  + a^\dagger_{\mathbf{k}\sigma })] \\
        & = \sum_{\mathbf{k},\sigma} \omega_\sigma^2(\mathbf{k} ) (a_{\mathbf{k}\sigma }a^\dagger_{\mathbf{k}\sigma } + a^\dagger_{\mathbf{k}\sigma } a_{\mathbf{k}\sigma })
    \end{aligned} 
\end{equation} 
如果我们令
\begin{equation}
    \begin{aligned} 
    b_{\mathbf{k}\sigma } = \sqrt{\frac{2\omega_\sigma(\mathbf{k} )}{\hbar}} a_{\mathbf{k}\sigma }
    \end{aligned} 
\end{equation}
这里我是不太严谨的,因为之前令 $ i\hbar =1$现在又加了个东西进来。所以推导还是有一些问题的.
并且我们可以算得对易关系
\begin{equation}
    \begin{aligned} 
    \relax [b_{\mathbf{k}\sigma },b^\dagger_{\mathbf{k'}\sigma '}] = \delta_{\mathbf{kk'} }\delta_{\sigma \sigma'}
    \end{aligned} 
\end{equation}
最后我们可以得到
\begin{equation}
    \begin{aligned} 
    H_1 &= \frac{1}{2}\sum_{\mathbf{k},\sigma} \hbar \omega_\sigma(\mathbf{k} ) (b^\dagger_{\mathbf{k}\sigma }b_{\mathbf{k}\sigma } 
    + b_{\mathbf{k}\sigma }b^\dagger_{\mathbf{k}\sigma }) \\
    &=\sum_{\mathbf{k},\sigma} [\frac{1}{2} \hbar \omega_\sigma (\mathbf{k} ) + \hbar \omega_\sigma b^\dagger_{\mathbf{k}\sigma }b_{\mathbf{k}\sigma }]
    \end{aligned} 
\end{equation}
综上所述,系统哈密顿量
\begin{equation}
    \begin{aligned} 
    H = \sum_{\mathbf{k}_0,\sigma }p^\dagger_{\mathbf{k}_0\sigma }p_{\mathbf{k}_0\sigma } + 
    \sum_{\mathbf{k},\sigma}' [ \hbar \omega_\sigma b^\dagger_{\mathbf{k}\sigma }b_{\mathbf{k}\sigma } +\frac{1}{2} \hbar \omega_\sigma (\mathbf{k} )]
    \end{aligned} 
\end{equation}
其中,等式右边第一项包含满足 $\omega_\sigma (\mathbf{k} ) = 0$的那些模式,第二项则包含 $\omega_\sigma (\mathbf{k} ) >0$的
那些模式。右边括号中的第二项是声场的零点振动能。但第一项不怎么关心,通常都是不写的,如果想要了解可以参考老师的笔记。忽略
第一项后
\begin{equation}
    \begin{aligned} 
        H = 
        \sum_{\mathbf{k},\sigma} [ \hbar \omega_\sigma b^\dagger_{\mathbf{k}\sigma }b_{\mathbf{k}\sigma } +\frac{1}{2} \hbar \omega_\sigma (\mathbf{k} )]
    \end{aligned} 
\end{equation}
通常称声场的场量子为声子, $b^\dagger_{\mathbf{k}\sigma }$和 $ b_{\mathbf{k}\sigma } $ 分别是声子的产生与湮灭
算子。由于声子的产生与湮灭算子满足 $[b_{\mathbf{k}\sigma },b^\dagger_{\mathbf{k'}\sigma '}] = \delta_{\mathbf{kk'} }\delta_{\sigma \sigma'}
$,声子属于玻色子。

\section{声子气体的热性质}
声子是玻色子,符合玻色统计,其统计密度为
\begin{equation}
    \begin{aligned} 
    \rho (H) = \frac{e^{-\beta H }}{ Tr (e^{-\beta H})}
    \end{aligned} 
\end{equation}
系统的内能为
\begin{equation}
    \begin{aligned} 
    U = \langle H \rangle = \sum_{\mathbf{k},\sigma}( \hbar \omega_\sigma (\mathbf{k} )\langle b^\dagger_{\mathbf{k}\sigma }b_{\mathbf{k}\sigma }\rangle +\frac{1}{2} \hbar \omega_\sigma (\mathbf{k} ))
    \end{aligned} 
\end{equation}
其中,
\begin{equation}
    \begin{aligned} 
        \langle b^\dagger_{\mathbf{k}\sigma }b_{\mathbf{k}\sigma }\rangle &=  Tr(b^\dagger_{\mathbf{k}\sigma }b_{\mathbf{k}\sigma } \rho(H)) \\
        &= \frac{1}{e^{\beta \hbar \omega_\sigma (\mathbf{k} )}-1}
    \end{aligned} 
\end{equation}
因此,内能为
\begin{equation}
    \begin{aligned} 
    U = \sum_{\mathbf{k},\sigma} \left( \frac{\hbar \omega_\sigma(\mathbf{k} )}{e^{\beta \hbar \omega_\sigma (\mathbf{k} )}-1}
    +  +\frac{1}{2} \hbar \omega_\sigma (\mathbf{k} ) \right)
    \end{aligned} 
\end{equation}
内能求出来了,就可以求比热。然后再复习一下Einstein模型和Debye模型,不过现在懒得弄了。
\section{长波方法---声学模}
\par 插播一个定义,如果 $\mathbf{k}= 0$,此时有
$\oemga_\sigma = 0$,那么,相应的声子频带称为声频支,相应的格波称为声学模;反之,如果 $\mathbf{k}= 0$时,
$\oemga_\sigma > 0$,相应的声子频带称为光频支,相应的格波称为光学模。
\par 我们现在考虑极低温情况,此时 $\omega (\mathbf{k}  \approx 0)$。现在不太想学,如果考试需要那就学一下
\section{长波方法---光学模}
要用到黄昆方程,如果考试需要,学一下吧
\section{态密度与范霍夫奇点}